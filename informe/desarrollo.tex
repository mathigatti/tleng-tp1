\section{Analizador léxico }

Para configurar el analizar léxico es necesario proveer expresiones regulares que reconozcan cada uno de los tokens. Para poder reconocer las palabras reservadas del lenguaje  fue necesario agregar lógica extra al analizador. Para evitar que las palabras reservadas sean usadas como variables se creo un diccionario donde cada elemento es el par (palabra reservado,token palabra reservada). Cuando una cadena del código siendo analizado se corresponde con la expresión regular del token VARIABLE, antes de aceptar el token se buscar la cadena en el diccionario. De encontrar la cadena en el diccionario se utiliza el tipo de token definido en el mismo, en caso contrario se devuelve un token VARIABLE. 
Como en etapas posteriores se verificara no solo la sintaxis, sino también que el código cumple con las restricciones de tipado del lenguaje, el lexer diferencia entre números, cadenas de caracteres y booleanos, devolviendo los tokens NUMERO,CADENA y BOOL respectivamente.

\subsection{Expresiones regulares para tokens}

\begin{verbatim}
                  DO ->    'do'
               WHILE ->    'while'
                 FOR ->    'for'
                  IF ->    'if'
                ELSE ->    'else'
                 RES ->    'res'
               RETURN ->    'return'
                BEGIN ->    'begin'
                  END ->    'end'
          CAPITALIZAR ->    'capitalizar'
               LENGTH ->    'length'
                PRINT ->    'print'
MULTIPLICACIONESCALAR ->    'multiplicacionescalar'
           COLINEALES ->    'colineales'
                  AND ->    'and'
                   OR ->    'or'
                  NOT ->    'not'
                MINUS ->    '-'
              ELEVADO ->    '^'
               MODULO ->    '%'
                  DIV ->    '/'
                MAYOR ->    '>'
                MENOR ->    '<'
                 PLUS ->    '+'
                TIMES ->    '*'
               LPAREN ->    '('
               RPAREN ->    ')'
            LCORCHETE ->    '['
            RCORCHETE ->    ']'
             LLAVEIZQ ->    '{'
             LLAVEDER ->    '}'
        INTERROGACION ->    '?'
                PUNTO ->    '.'
            DOSPUNTOS ->    ':'
           PUNTOYCOMA ->    ';'
                 COMA ->    ','
                IGUAL ->    '='('\n')*'='
             DISTINTO ->    '!'('\n')*'='
              AGREGAR ->    '+'('\n')*'='
                SACAR ->    '-'('\n')*'='
               DIVIDI ->    '/'('\n')*'='
              MULTIPL ->    '*'('\n')*'='
               MASMAS ->    '+'('\n')*'+'
           COMENTARIO ->    '#'.*
               CADENA -> '"' .*? '"'
                 BOOL ->    'true' | 'false' | 'FALSE' | 'TRUE'
             VARIABLE -> ([a-z][A-Z])([a-z][A-Z]|'_'|[0-9])*
           ASGINACION ->   '='

\end{verbatim}
\section{Análisis sintáctico}

Se procedió a crear una gramática que pudiese reconocer el lenguaje SLS. 
La herramienta utilizada para construir el analizador sintáctica,a la cual se le proveen los tokens del lexer y las producciones de la gramática, crea un analizador LALR. 

La creación de la gramática fue un proceso iterativo. Al procesar las primeras versiones de la gramática con el ply se encontraban gran número de conflictos. Se fue modificando la gramática, resolviendo ambigüedades para eliminar conflictos shift/reduce y reduce/reduce encontrados, lo que hizo que la misma creciera en cantidad de producciones y en complejidad.
Una vez eliminados los conflictos que no podían ser solucionados seleccionando un acción  de la tabla LALR, se procedió a ejecutar tests mas complejos provistos por la cátedra para terminar de pulir los últimos detalles. Dado que la gramática es ambigua no es LR(1) y por ende no es LALR. 
    En las siguientes secciones se presentan las producciones de la gramática, solución de conflicto de if-else mediante selección de acción y chequeo de tipos. 
    

\subsection{Producciones gramática}\label{sec:pgram}
\begin{verbatim}
    <programa>      :   <sentencia> <programa'>   
                    |   <control> <programa'>   
                    |   COMENTARIO <programa>   

    <programa'>     :   <sentencia> <programa'>   
                    |   <control> <programa'>   
                    |   COMENTARIO <programa'>   
                    |   <empty>

    <sentencia>     :   <var_asig> ';'
                    |   <funcion> ';'

    <control>       :   <ifelse>
                    |   <loop>

    <control_cond>  :   <var_asig_l>
                    |   <exp_bool>
                    |   <comparacion>
                    |   <op_ternario>

    <loop>          :   'while' '(' <control_cond> ')' <bloque>
                    |   'do' <bloque> 'while' '(' <control_cond> ')' ';' 
                    |   <for>

    <for>           :   'for' '(' <for_term> ',' <form_term_2> ',' <for_term> ')' 
                                <bloque>


    <for_term>      :   <var_asig> 
                    |   <empty>

    <form_term_2>   :   <valores>
                    |   <comparacion>

    <ifelse>        :   'if' '(' <control_cond> ')' <bloque>
                    |   'if' '(' <control_cond> ')' <bloque> else <bloque>

    <bloque>        :   COMENTARIO <bloque>
                    |   <sentencia>
                    |   <control>
                    |   '{' <programa'> '}'

    <funcion>       :   func_ret
                    |   func_void

    <func_void>     :   'print' '(' <valores> ')'

    <func_ret>      :   <func_ret_int>
                    |   <func_ret_cadena>
                    |   <func_ret_bool>
                    |   <func_ret_arreglo>

<func_ret_arreglo>  :   'multiplicarEscalar' '(' <valores> ',' <valores> ')'
                    |   'multiplicarEscalar' '(' <valores> ',' <valores> ',' 
                                                 <valores>')'
                    
<func_ret_bool>     :   'colineales' '(' <valores> ',' <valores> ')'

<func_ret_cadena>   :   'capitalizar' '(' <valores> ')'

    <func_ret_int>  :   'length' '(' valores ')'


    <valores>       :   <exp_arit>
                    |   <exp_bool>
                    |   <exp_cadena>
                    |   <exp_arreglo>
                    |   <registro>
                    |   <registro> '.' VARIABLE
                    |   <var_asig_l>
                    |   <op_ternario>

    <exp_arreglo>   :   '[' <list_valores> ']'
                    |   '[' <list_valores> ']' <exp_arreglo>
                    |   '['']'
                    |   func_ret_arreglo

    <lista_valores> :  <valores>
                    |  <valores> ',' <lista_valores>

    <registro>      :   '{' <reg_item> '}'

    <reg_item>      :   VARIABLE ':' <valores> ',' <reg_item>
                    |   VARIABLE ':' <valores> 

    <var_asig_l>    :   VARIABLE
                    |   RES
                    |   VARIABLE <var_member>


    <var_member>    :   '[' var_asig_l ']' <var_member>
                    |   '[' <exp_arit> ']'  <var_member>
                    |   '.' VARIABLE <var_member>
                    |   '[' <exp_arit> ']' 
                    |   '[' <var_asig_l> ']' 
                    |   '.' VARIABLE 


    <var_asig>      :   <var_asig_l> '++'
                    |   '++' <var_asig_l>
                    |   <var_asig_l> '--'
                    |   '--' <var_asig_l>
                    |   <var_asig_l> '*=' <valores>
                    |   <var_asig_l> '/=' <valores>
                    |   <var_asig_l> '+=' <valores>
                    |   <var_asig_l> '-=' <valores>
                    |   <var_asig_l> '=' <valores>
                    |   <var_asig_l> '=' <comparacion>
                    |   <var_asig_l> '=' <operador_ternario>

    <op_ternario>   :   <valores> '?' <valores> ':' <valores>
                    |   <comparacion> '?' <valores> ':' <valores>
                    |   <valores> '?' <valores> ':' <op_ternario>
                    |   <comparacion> '?' <valores> ':' <op_ternario> 
                
    <var_oper>      :   <var_asig_l>
                    |   '(' <op_ternario> ')'
                    |   <exp_arreglo>
                    |   <registro> '.' VARIABLE 

 
    <exp_arit>      :   <exp_arit> '+' <term>
                    |   <exp_arit> '+' <var_oper>
                    |   <var_oper> '+' <term>
                    |   <var_oper> '+' <var_oper>
                    |   <exp_arit> '-' <term>
                    |   <exp_arit> '-' <var_oper>
                    |   <var_oper> '-' <term>
                    |   <var_oper> '-' <var_oper>
                    |   <term>


    <arit_oper_2>   :   '*'
                    |   '/'
                    |   '%'
    
    <term>          :   <term>  <arti_oper_2> <factor>
                    |   <var_oper> <arit_oper_2> <factor>
                    |   <term> <arit_oper_2> <var_oper>
                    |   <var_oper> <arit_oper_2> <var_oper> 
                    |   <factor>

    <factor>        :   <base> '^' <sigexp>
                    |   <var_oper> '^' <sigexp>
                    |   '-' <base>  
                    |   '-' <var_oper>  
                    |   <base> '++'
                    |   <base> '--'
                    |   '++' <base>
                    |   '--'<base>
                    |   <var_oper> '++'
                    |   '++' <var_oper>
                    |   <var_oper> '--'
                    |   '--' <var_oper>
                    |   <base>

    <base>          :   '(' <exp_arit> ')'
                    |   '(' <var_oper> ')'
                    |   NUMBER
                    |   <func_int>

    <sigexp>        :   '-' <exp>
                    |   <exp>

    <exp>           :   <var_oper>
                    |   NUMBER
                    |   '(' <exp_arit> ')'

    <exp_cadena>    :   <exp_cadena> '+' <term_cadena>
                    |   <var_oper> '+' <term_cadena>
                    |   <term_cadena> '+' <var_oper> 
                    |   <term_cadena> 

    <term_cadena>   :   CADENA
                    |   <func_ret_cadena>
                    |   '(' <exp_cadena> ')'



    <exp_bool>      :   <exp_bool> AND <term_bool>
                    |   <var_oper> AND <term_bool>
                    |   <exp_bool> AND <var_oper>
                    |   <var_oper> AND <var_oper>
                    |   <term_bool>

    <term_bool>     :   <term_bool> OR <factor_bool>
                    |   <var_oper> OR <factor_bool>
                    |   <term_bool> OR <var_oper>
                    |   <var_oper> OR <var_oper>
                    |   'not' <factor_bool>
                    |   'not' <var_oper>

    <factor_bool>   :   BOOL
                    |   '(' <exp_bool> ')'
                    |   '(' <comparacion> ')'
                    |   <func_bool>

    <op_comp>       :   '=='
                    |   '>'
                    |   '<'
                    |   '!='

    <comparacion>   :   <valores> <op_comp>  <valores>

    
\end{verbatim}
\subsection{Resolución de conflicto if-else}\label{sec:pgram}
Nuestra gramática presenta la ambigüedad conocida como "dangling else". En nuestra gramática una sentencia de la forma "if E1 B1 else if E2 B2 else B3" tiene dos arboles de parseo posibles cuando <bloque> deriva en <sentencia> , ver figura ~\ref{fig:trees} .  En este caso se opto por resolver el conflicto seleccionando la acción shift de la tabla, dado que con dicha selección se logra el comportamiento deseado de hacer matchear el "else" con el anterior if sin matchear.  

\begin{figure}[H]

    \begin{subfigure}[b]{0.3\textwidth}
    \begin{tikzpicture}[every node/.style={align=center},level distance=1.5cm]
    \Tree [ .<ifelse> \textbf{if} [ .<control\_cond> \edge[roof]; {E1}  ] [ .<bloque> [ .<sentencia> \textbf{if} [ .<control\_cond> \edge[roof]; {E2}  ]  ] [ .<bloque> [ .<sentencia> \edge[roof]; {B1}  ] ] ] \textbf{else}  [ .<bloque> [ .<sentencia> \edge[roof]; {B2}  ] ] ] 
    \end{tikzpicture}
    \end{subfigure}
    
    \begin{subfigure}[b]{0.3\textwidth}
    \begin{tikzpicture}[every node/.style={align=center},level distance=1.5cm]
    \Tree [ .<ifelse> \textbf{if} [ .<control\_cond> \edge[roof]; {E1}  ] [ .<bloque> [ .<sentencia> \textbf{if} [ .<control\_cond> \edge[roof]; {E2}  ] [ .<bloque> [ .<sentencia> \edge[roof]; {B1}  ] ] \textbf{else}  [ .<bloque> [ .<sentencia> \edge[roof]; {B2}  ] ] ] ] ]
    \end{tikzpicture}
    \end{subfigure}
    



     \caption{Versión simplificada de arboles de parseo para \\"if E1 B1 else if E2 B2 else B3 }
      \label{fig:trees}
\end{figure}    
\subsection{Chequeo de tipos}

Unos de los objetivos de la implementación era verificar que el código cumpliese con restricciones de tipado del lenguaje. Parte de este chequeo se hace a través de la sintaxis no permitiendo  que se formen cadenas que violen dichas restricciones. 
Por ejemplo las producciones mostradas a continuación del caso  base de exp\_arit no aseguran que los tokens BOOL y CADENA no van a aparecer en una expresión aritmética. 
\begin{verbatim}
    <base>          :   '(' <exp_arit> ')'
                    |   '(' <var_oper> ')'
                    |   NUMBER
                    |   <func_int>

\end{verbatim}
	Ahora bien, no es posible chequear con esta gramática expresiones donde se usan variables . Para ellos se extendió la gramática a una una gramática de atributos para poder poder hacer una validación más completa. Esto se logra mediante la síntesis del atributo tipo y chequeando que se cumplan las restricciones en el momento que se utilizan las variables. Esta extensión no solo permitió hacer un chequeo mas completo sino adema permitió simplificar algunas producciones por ejemplo el de las comparaciones donde las producciones:
    
\begin{verbatim}
<comparacion>   :   <exp_cadena> <op_comp>  <exp_cadena>
                 |   <var_oper> <op_comp>  <exp_cadena>
                 |   <exp_cadena> <op_comp> <var_oper> 
                 |   <exp_arit>   <op_comp>   <exp_arit>
                 |   <var_oper>   <op_comp>   <exp_arit>
                 |   <exp_arit>   <op_comp>   <var_oper>
                 |   <var_oper>   <op_comp>  <var_oper>
                 |   <exp_bool> '==' <exp_bool>
                 |   <exp_bool> '!=' <exp_bool>
\end{verbatim}
fueron reemplazadas por una sola producción:
\begin{verbatim}
 <comparacion>   :   <valores> <op_comp>  <valores>
\end{verbatim}
 
Otro caso en el que la síntesis del atributo tipo resulto útil fue para expresiones que son suma de todas variables. Dicha expresión es válida tanto para cadenas como para números, pero solo se puede derivar a partir de <exp\_arit>. Utilizando el atributo de tipo, podemos permitir que dicha expresión se forme tanto para suma de cadenas como para suma de variables con números.
    En la siguiente sección se profundiza sobre el análisis semántico.

\section{Análisis semántico}

Con lo descrito hasta ahora podemos reconocer y aceptar textos escritos con una sintaxis correcta bajo las reglas enunciadas en las consignas del trabajo pero hay posibles expresiones que no queremos que sean aceptadas a pesar de que cumplan las reglas de la gramática, porque pueden carecer de significado para el proposito del lenguaje. Para esta situación extenderemos lo que tenemos a una gramática de tipos. De esta manera podremos detectar, por ejemplo, cuando se intenta realizar un cálculo aritmetico con una variable que no fue declarada.


\subsection{Introduccion}

Para hacer esto utilizamos algunas estructuras de datos que nos parecieron adecuadas para este fin. La principal es una tupla [STRING, TIPO] la cual nos proporciona los dos datos fundamentales con los que trabaja nuestro programa, tiene al texto de salida como primer componente y como segunda componente su tipo. Por ejemplo en el caso de que el texto de entrada sea un $INT$ o una $CADENA$ les asignaremos dichos valores como tipo por otro lado en el caso de las variables esto no es tan simple, ya que pueden no tener tipo o ir cambiandolo a medida que reciba nuevas asignaciones. Para manejar esta situación decidimos que si una variable no esta definida su tipo será 'ND'. Si en algún momento se le asigna un valor entonces almacenaremos su nuevo tipo en un diccionario llamado $variables\_dict$. Entonces al recorrer las reglas de la gramática que forman el texto iremos armando el texto de sálida al mismo tiempo que verificaremos que no haya ningún error semántico. Como ejemplo se puede ver como se describe el tipo de la siguiente regla.

\begin{verbatim}

def p_factor_var_op_pp(p):
    'factor : var_oper MASMAS'
    p[0] = [p[1][0] + '++', p[1][1]]

    if not esNumber(p[1][1]):
        pass
        raise SemanticException('ERRORTIPO',p.lineno(1),p.lexpos(1))

\end{verbatim}

El valor del tipo de factor es $p[0]$ el cual como dijimos queda definido por una tupla, la primer componente es el valor de la variable $var\_oper$ seguido del signo $++$, el segundo es el tipo, el cual dependerá unicamente del tipo de $var\_oper$, al cual se accede escribiendo $p[1][1]$. En este caso sería un problema que la variable no fuera de tipo $INT$ o $FLOAT$, para verificar esto se utiliza la funcion $esNumber$ y dependiendo su resultado se expondrá un mensaje de error de tipo o no.

\subsection{Registros y Arreglos}

Otro posible error semántico es el de los registros y el acceso a variables que no pertenecen al mismo. Para esto, de manera muy similar al caso anterior, utilizamos un diccionario llamado $reg\_dic$. Aquí se guardan todas las variables de cada registro que se define en el código de entrada. Y se verifica en caso de haber un intento de acceder al valor de la variable de algún registro si la misma fue definida. En la siguiente regla se ve un caso en el que esto ocurre.

\begin{verbatim}

def p_valores_reg2(p):
    'valores : reg PUNTO VARIABLE'
    p[0] = [p[1][0] + '.' + p[3], 'CUALQUIER_TIPO']
    if estaReg(p[3]) == 'ND':
        pass
        raise SemanticException('REGISTRO',p.lineno(1),p.lexpos(1))


\end{verbatim}

Aquí verificamos utilizando la función $estaReg$ si $VARIABLE$ fue definida como clave de algún registro. De no ser así devolvemos un error de tipo $'REGISTRO'$. Notar que para no complicar demasiado el código decidimos no verificar cual es el tipo exacto de esa variable en el registro. Por eso le asignamos como tipo $'CUALQUIER\_TIPO'$ y en estos caso permitimos que esta sea parte de cualquier tipo de expresiones, ya sea booleana, aritmetica o cadena. Lo mismo sucede cuando accedemos al valor de un arreglo.

\subsection{Comentarios}

Para reconocer si un comentario es inline o no utilizamos la funcion $lineno$ la cual nos proporciona la linea en la cual esta escrita una sentencia, si dicha sentencia esta escrita en la misma que un comentario, luego a la salida imprimimos ambos a la misma altura, en caso contrario los separamos en distintas lineas. Un ejemplo de como hacemos esto se puede ver en el siguiente fracmento de codigo extraido de $parser\_rules.py$ .

\begin{verbatim}

def p_programa_s_pp(p):
    'p : sentencia pp'
    if (p.lineno(1) == p.lineno(2)) and p[2][1] == 'COMENTARIO':    	
        p[0] = [p[1][0] + ' '  + p[2][0], 'ND']
    else:
        p[0] = [p[1][0] + '\n'  + p[2][0], 'ND']

\end{verbatim} 

El if verifica que ambas expresiones esten en la misma linea como dijimos pero a parte, obviamente, que el tipo de la segunda expresión sea $'COMENTARIO'$, de no ser así, deberiamos imprimir ambas sentencias en distintas lineas por mas que originalmente hubieran estado juntas.

\subsection{Tabulaciones}

Para manejar las tabulaciones usamos la función $find\_and\_replace$ que recibe como entrada un string y la agrega a todos los saltos de linea una tabulacion. De esta manera cuando sea necesario, por ejemplo en los bloques de los while, for e if's llamarremos a esta función para que les agregue un espaciado extra.

\begin{verbatim}

def p_ifSinElse(p):
    'ifelse : IF LPAREN control_cond_term RPAREN bloque'
    p[0] = ['If(' + p[3][0] + ')\n    ' + find_and_replace(p[5][0]) + '\n', 'ND']

\end{verbatim}

\subsection{Detalles Finales}

Como el código esta comentado y pensamos que será de fácil lectura ya que no tiene mayores dificultades no extenderemos demasiado esta sección, pero creemos que vale la pena mencionar algunos detalles finales para facilitar el entendimiento del lector. El valor 'ND' es un poco ambigüo ya que lo utilizamos mayormente para simbolizar el tipo de una variable que no esta definida pero también en algunas partes está asignado al tipo de algunas expresiones las cuales no necesitamos chequear nunca. Los posibles errores semánticos son variados, se puede encontrar el listado de todos los errores posibles en el archivo $semantic\_error.py$
