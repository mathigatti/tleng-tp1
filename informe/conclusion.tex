En primer instancia, pudimos crear muy rápidamente una gramática que pedía todo lo que el lenguaje especificado nos pedía, 
pero contenía muchísimas ambiguedades, ambiguedades que si bien fueron muchas y con gran número de producciones, aplicando
metódica y ordenadamente el algoritmo para desambiguar, nos quedo una gramática no ambigua, con muy pocos conflictos, que 
se podían resolver para que pueda un parser LALR parsearla.

Sin embargo, a medida que que desambiguamos la gramática la cantidad de producciones aumenteaba significativamente haciendola más 
compleja de entender y difícil de modificar.

A veces es preferible resolver un conflicto solucionando una accion, y evitar complejizar la grámatica al desambiguar.
En cuanto al chuqueo de tipos a fin de mantener un gŕamatica relativamnete pequeña y manjeable es preferible delegar 
todo lo posible al análisis semántico. 

Muchas cosas se pueden restringir tanto semántica como sintácticamente, en nuestro caso tratamos de que todo lo que podíamos
resolver con el anális semántico se resolviera ahí, para que no nos genere coflictos ni ambiguedades en una gramática demasiado
extensa. 

Por ejemplos casos de operadores de comparación, que esperan dos expresiones del mismo tipo, claramente es mucho más fácil poder
restringirlo del lado de la semántica, con algo de código que agregar varias producciones más que parseen correctamente estos casos.
También nos enfrentamos a casos en los que necesitabamos si o si poder hacerlo desde el lado semántico, facilitando así el parseo
o casos en los que debimos hacerlo desde el lado sintáctico. 


     
