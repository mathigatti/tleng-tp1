A continuación presentaremos una serie de casos correctos e incorrectos respecto a lo descrito en las cosignas del trabajo práctico. Cada test tiene como propósito corroborar el buen funcionamiento de los distintas partes de nuestro programa. Intentamos poner a prueba la capacidad del parser para discernir entre un texto con buena sintaxis y uno mal escrito, también con el lexer probamos que se excluyan las palabras reservadas y por último que los posibles errores semánticos sean detectados.

A parte de estos tests, utilizamos algunos mas creados por nosotros y tambien los proporcionados por la catedra. Nuestro programa funciona correctamente para todos ellos.

\subsection{Casos Correctos.}
\subsubsection{Caso 1:}

\begin{verbatim}


#NADA
z = [a.edad + length("doce"+capitalizar("2")) / 2, 1.0 
+ (siete ? ((aa OR bb AND (2!=length(""))?1:2)):
g*12 %3) , NOT aa ? b / -5 : c]; while(dos);

a = 0;
b = 1;
a = b;
c = 2.5;
d = "Hola";
e = d;

f = [1, 2, 3];
a = f[1];
b = f[2];

g = [1, a, 2, b, 3];
g[a] = b;
g[b] = a;

h = [f[g[a]], g[g[b]]];
h[f[g[a]]] = g[f[g[g[b]]]];


\end{verbatim}

\subsubsection{Caso 2:}

\begin{verbatim}
for (i=0; i<10; i++) {
while(true) {
for(;true;) {
do {
a=0;
}
while(false);
}
}
}

\end{verbatim}

\subsubsection{Caso 3:}

\begin{verbatim}

a = -10;
b = 2 + 3 - 8 + 1;
c = 3 * 5 % 4 / 6;
d = 2.0 ^ 9.0;
e = -3 * (4 / ((3) + 8 ^ ((1))) - -5 + -(7) * 6);

f = -a * (b / ((a) + d ^ ((c))) - -b + -(c) * a);

g = [1.5, 2 * (7) + 5 / 0, a * d, 8 ^ 1.0, (6)
 / (6), (3 + 3) + 0.0, e - 5, 2 * -f];
h = [1,b,3];

g[2 * 4] = 10 + (3 * 2/1);
g[b ^ b] = (g[7 * (5) + (h[(2)])]);
g[9 - (5 % 3)] = g[b % h[b]];

while (true) {
	d = 1;
	a = 0;
}

do {
	# Co
	# men
	# tario
	b = 2;
	c = 3;
} while ((d > 10) AND NOT false);

usuario = {nombre:"Al", edad:50};
print(capitalizar(usuario.nombre));
nacimiento = 2016;
nacimiento -= usuario.edad;
usuarios = [{nombre:"Mr.X", edad:10}, usuario];

suma = 0;
for (i = 0; i < length(usuarios); i++) {
	print(usuarios[i].nombre);
	suma += usuarios[i].edad;
}


\end{verbatim}


\subsection{Casos incorrectos.}
\subsubsection{Caso 1.}

\begin{verbatim}
else
  b=0;

\end{verbatim}

\subsubsection{Caso 2.}

\begin{verbatim}
while(true){
#comentario sin sentencias
}

\end{verbatim}

\subsubsection{Caso 3.}

\begin{verbatim}
for(true){}
\end{verbatim}

\subsubsection{Caso 4.}

\begin{verbatim}
While = 10;
\end{verbatim}

\subsubsection{Caso 5.}

\begin{verbatim}
a = 5
\end{verbatim}

\subsubsection{Caso 6.}

\begin{verbatim}
multiplicacionEscalar("1,2,3");
\end{verbatim}

\subsubsection{Caso 7.}

\begin{verbatim}
multiplicacionEscalar();
\end{verbatim}

